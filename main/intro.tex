\part*{Introduction}
\addcontentsline{toc}{part}{Introduction}
\markboth{Introduction}{Introduction}

\vspace*{\fill}
\epigraph{J’entendais tout cela, moi ; car nous autres esclaves, nous sommes doués contre nos maîtres d’une pénétration.}{\textit{Cléanthis, L'île des esclaves, 1725}}

\vfill\clearpage
%abrupt

Au cours du dernier siècle notamment, s’est cristallisée l’idée que l’on assistait à une montée de l'« individualisme », tendance culturelle censée privilégier la valeur et les droits de l'individu par rapport à ceux de la société. De fait, l'individualisme devient dès la seconde moitié du XIX\ieme ~siècle un prisme d’analyse méthodologique à part entière pour la sociologie naissante. De là, le terme semble peu à peu acquérir une connotation négative : il n’est aujourd’hui pas rare qu’il soit apparenté à la dénonciation d’une focalisation excessive sur le soi, un égocentrisme contemporain qui proliférerait au détriment de notre capacité à prêter attention aux autres\footnote{\cite{twenge_generation_2006}}\footnote{\textit{The Century of the Self}, Documentaire BBC : \url{https://www.youtube.com/watch?v=jymMjNc0igI}}.

Or, cette notion portée aux nues par la sociologie-même semble ne pouvoir révéler de manière plus limpide les limites des sciences humaines et sociales « traditionnelles » (du moins dans leur penchant holiste) s’agissant de la compréhension de tels mouvements culturels. Comment, de fait, la tendance à donner la priorité aux individus sur le groupe, contre sa propre cohésion, pourrait-elle s’expliquer en référence à ce même groupe ? Pourquoi le capitalisme, et son pendant psychologique que serait l’individualisme, émergent-t-ils à la fin du XVII\ieme ~siècle ? Et pourquoi pas avant ? Ni plus tard ? Comment, en fond, expliquer un fait social par un autre fait social, pour reprendre l'adage durkheimien, et où s’arrête une telle tautologie ?

Tout d’abord, il faut remarquer que l’on trouve la mention d’idées analogues au concept d’individualisme bien avant l’instauration du système capitaliste, ce dès le Moyen-Âge\footnote{\cite{payen_humanisme_1985}}\footnote{\cite{giddens_modernity_1991}}\footnote{\cite{gourevitch_naissance_1997}}\footnote{\cite{rosenwein_barbara_h_and_c_s_heppleston_y_2005}}. Phénomène non plus spécifiquement moderne, qu’est-ce qui explique, dès lors, que certaines périodes historiques, ou bien encore certaines sociétés, puissent être considérées comme plus individualistes que d’autres ? 

\bigskip

L’anthropologie évolutive s’attache à étudier la diversité des comportements humains dans le contexte de leur histoire évolutive. En particulier, la psychologie « évolutionniste », « évolutive » ou « évolutionnaire », s’intéresse à nos mécanismes de pensée, appréhendés à partir de la théorie de l’évolution biologique. Elle fait de notre environnement et de ses implications pour le fitness de chaque individu, à savoir son succès reproductif, un facteur explicatif central de notre psyché. Sur cette base, elle propose ainsi d’expliquer, ultimement, les modifications des préférences psychologiques humaines qui semblent intervenir au fil de notre Histoire par des variations du niveau de ressources offertes par notre milieu.

Cette précédence périodiquement donnée à l’individu sur le groupe, dans certaines sociétés, pourrait donc être réductible à des considérations matérielles, entendues au sens large (en incluant par exemple l’existence d’institutions). Encore faut-il définir ce qu’est l’individualisme du point de vue de la psychologie individuelle : quel(s) mécanisme(s) de pensée, précisément, pourrai(ent) motiver ces comportements que l’on qualifie synthétiquement d’ « individualistes » ?  Il semble que la notion d’ « agentivité » soit la plus à même de remplir ce rôle. Quoique définie de façon disparate, cette dimension psychologique fait globalement référence à la capacité d’un individu, réelle ou telle qu’elle est perçue par ce dernier, à agir sur le monde qui l’entoure, ce dans le but de poursuivre des objectifs propres. 

En dehors de toute considération hiérarchique, Vallacher\footnote{\cite{vallacher_levels_1989}} développe notamment l’idée que l’on pourrait placer chaque individu le long d’un continuum agentif, selon qu’il envisage le plus souvent ses actions à l’aune de leurs détails mécaniques (faiblement agentif), ou plutôt de significations plus abstraites (hautement agentif).

Pour mesurer un tel degré d’agentivité, l’auteur fait appel au langage : dans son étude, les participants doivent choisir, parmi deux formulations, celle qui décrit le mieux, selon eux, l’activité en question. A titre d’exemple, l’action « nettoyer la maison » (\textit{cleaning the house} en anglais) est alternativement décrite comme le fait de « faire état de sa propre hygiène » (\textit{showing one’s cleanliness}) ou bien simplement de « passer l’aspirateur » (\textit{vaccuming the floor}). Choisir la première formulation est censée dénoter relativement plus d’agentivité qu’opter pour la seconde.

Là où l’auteur ne s’aventure pas, c’est dans la possibilité de systématiser une telle mesure. L’idée que l’utilisation d’un lexique spécifique puisse être corrélée à la possession de certains traits psychologiques n’est pas nouvelle (voir la section I.2.2 \textit{Individualisme et pronoms: même méthode, nouveau raisonnement}), y compris s’agissant d’agentivité. Néanmoins, elle n’a que très rarement été étendue, outre le choix d'un lexique spécifique, à la mobilisation de structures grammaticales ou syntaxiques particulières, qui pourraient s’avérer préférentiellement associées à un état d’esprit plus agentif.

Ce mémoire s’attache donc à déterminer si l’élaboration d’un indice global d’agentivité, construit par agrégation de composantes linguistiques diverses, aussi bien lexicales que « structurelles », est envisageable. Si tel est le cas, l’objectif ultime de cette recherche consistera à mesurer le « score » agentif obtenu par différentes œuvres publiées dans une société donnée au fil du temps, afin de tester l’hypothèse selon laquelle ce score corrèle avec les variations du niveau de ressources de la société en question.

Pour ce faire, nous avons opté pour l’analyse de textes littéraires. Ceux-ci présentaient l’avantage de constituer un corpus conséquent, couvrant une période très large, tout en étant fournis en formules linguistiques suffisamment élaborées. En outre, l’on suppose ici, toujours selon notre paradigme évolutionnaire, que les auteurs de ces œuvres, quoique fictionnelles, se doivent toujours de répondre \textit{a minima} aux attentes de leur public. Ils parleraient ainsi, plus ou moins consciemment, un langage propre à évoquer la psychologie de toute une époque.