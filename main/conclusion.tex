\part*{Conclusion}
\addcontentsline{toc}{part}{Conclusion}
\markboth{Conclusion}{Conclusion} 

\begin{quote}
\textit{ « Nous autres esclaves, nous sommes doués contre nos maitres \\ d’une pénétration »}
\end{quote}

Ce que Marivaux confine à une relation entre une esclave et son maitre, qui obligerait la première à un effort d’analyse des attitudes et du langage de celle qui la soumet dans la seule intention de ne pas s’y laisser prendre, et sûrement de prévenir quelques mauvais traitements, pourrait ainsi valoir beaucoup plus globalement. « Nous sommes doués \textit{comme} nos maitres d’une pénétration », voilà peut-être ce qu’aurait pu dire, à la place, le personnage esclave ou serviteur du XVIII\ieme, et dévoilant par-là l'état d'esprit de toute une époque. C’est du moins la thèse soutenue dans ce mémoire : celle d’un nivellement par le haut du degré d’agentivité individuel au cours des derniers siècles, déterminé par l’augmentation du niveau des ressources environnantes.

Pour espérer pouvoir retracer et fonder une telle évolution, il nous fallait passer par un support objectif : le langage s’est naturellement imposé comme un terrain d’analyse privilégié. 

A travers l’élaboration d’un indice linguistique d’agentivité global, il s’avère finalement possible de prédire, avec une assez grande précision, le statut des locuteurs à l’origine des répliques analysées dans une pièce donnée. Si cet indice manque encore de validité externe, il est un premier pas vers une compréhension plus fine des structures linguistiques profondes, grammaticales et syntaxiques (et outre l’utilisation de pronoms personnels à la première personne), qui pourraient bien soutenir notre discours à l’heure d’un « individualisme » exacerbé, notion intimement liée à celle d’agentivité. Et par là-même, il ouvre la porte à une étude plus approfondie des grands changements psychologiques de l’histoire humaine.

Bien entendu, si notre paradigme fait de ce degré croissant d’agentivité, ou d'individualisme tel que nous l’avons redéfini, une tendance purement mécanique, portée par l’amélioration du confort de vie, cela n’exclut pas qu’il puisse être devenu  « mal-adapté » à notre environnement actuel, au sens proprement évolutionnaire du terme. Comme mentionné dans la partie \textit{I. 2.2. Individualisme et pronoms: même méthode, nouveau raisonnement}, l’augmentation apparente des troubles neurotiques sur la période récente\footnote{\cite{twenge_age_2000}} pourrait en ce sens être interprété comme symptomatique d’une stratégie d’identification  « haut niveau » dont la quasi-permanence serait devenue sous-optimale dans nos sociétés modernes. En feignant d’ignorer ici des inégalités sociales bien entendu non négligeables, il semble en effet que tout un chacun dispose aujourd'hui, en Occident du moins, d’un niveau de ressources suffisant pour s’engager dans une interprétation particulièrement abstraite et distale de ses actions. Le moindre acte entrepris deviendrait alors le théâtre d’une délibération identitaire qu’il semble difficile de pouvoir supporter au quotidien sans risquer de se perdre, bien loin de l'agir, dans les méandres du moi. Et c’est peut-être ce que la notion d’individualisme, dans son sens péjoratif, pointe précisément du doigt. De là l’importance, également, d’études psycholinguistiques qui devraient venir compléter ces analyses dans un effort de triangularisation des données, et qui pourraient envisager y trouver, à terme, des pistes de diagnostic clinique par le biais du langage. 

