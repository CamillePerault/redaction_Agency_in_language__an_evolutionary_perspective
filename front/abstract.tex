\section*{Résumé}
\addcontentsline{toc}{chapter}{Résumé}
Ce projet interroge la possibilité de détecter de façon cohérente et systématique, dans le langage écrit, une des grandes dimensions psychologiques humaines : celle de l’« agentivité ». En s’intéressant aux ouvrages fictionnels, en particulier les pièces de théâtre pour leur forme stylistique relativement standardisée et leurs personnages types, il entend \textit{in fine} faire le pont entre connotation agentive d’un texte et le degré d’agentivité de son auteur ainsi que de ses contemporains, ce dans une perspective évolutionnaire. Etant donné le caractère très exploratoire de ce travail, et l’ambiguïté de la notion-même d’agentivité étudiée ici, la majeure partie de la recherche a consisté, en amont, à définir plus finement la dimension psychologique en question, et, sur cette base, les marqueurs linguistiques théoriquement à même de refléter le niveau d’agentivité de leurs locuteurs. Ceci explique le relatif déséquilibre entre les chapitres théoriques et ceux ayant trait aux résultats, une grande partie des analyses statistiques restant à mener.


\medskip

\textbf{Mots-clés: agentivité ; psycholinguistique ; anthropologie évolutive ; théâtre classique ; humanités numériques ; lecture distante ; traitement automatique de la langue}

\textbf{Informations bibliographiques:} Camille Perault, \textit{Détecter et quantifier l'agentivité dans la langue : une perspective évolutionnaire}, mémoire de master 2 \og Humanités Numériques\fg{}, dir. [Nicolas Baumard, Jean-Baptiste Camps], Université Paris, Sciences \& Lettres, 2023.


\section*{Abstract}
\addcontentsline{toc}{chapter}{Abstract}
This project investigates the possibility of consistently and systematically detecting, in written language, one of the great human psychological dimensions: that of "agency". By focusing on fictional works, particularly plays for their relatively standardized stylistic form and their typical characters, it ultimately intends to bridge the gap between the agentive connotation of a text and the degree of agency of its author and his or her contemporaries, this in an evolutionary perspective. Given the highly exploratory nature of this work, and the ambiguity of the very notion of agency studied here, the bulk of the research has consisted, upstream, in defining more precisely the psychological dimension in question, and, on this basis, the linguistic markers theoretically capable of reflecting the level of agency of their speakers. This explains the relative imbalance between the theoretical chapters and those dealing with results, with a large proportion of the statistical analyses still to be carried out.

\medskip

\textbf{Keywords: agency ; psycholinguistics ; evolutionary anthropology ; classical theater ; digital humanities ; remote reading ; automatic language processing}

\textbf{Informations bibliographiques:} Camille Perault, \textit{Detecting and quantifying agentivity in language: an evolutionary perspective}, Master 2 thesis \og Digital Humanities\fg{}, dir. [Nicolas Baumard, Jean-Baptiste Camps], Université Paris, Sciences \& Lettres, 2023.


\clearpage